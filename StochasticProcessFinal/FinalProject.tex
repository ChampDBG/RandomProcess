\documentclass[]{article}
\usepackage{lmodern}
\usepackage{amssymb,amsmath}
\usepackage{ifxetex,ifluatex}
\usepackage{fixltx2e} % provides \textsubscript
\ifnum 0\ifxetex 1\fi\ifluatex 1\fi=0 % if pdftex
  \usepackage[T1]{fontenc}
  \usepackage[utf8]{inputenc}
\else % if luatex or xelatex
  \ifxetex
    \usepackage{mathspec}
  \else
    \usepackage{fontspec}
  \fi
  \defaultfontfeatures{Ligatures=TeX,Scale=MatchLowercase}
\fi
% use upquote if available, for straight quotes in verbatim environments
\IfFileExists{upquote.sty}{\usepackage{upquote}}{}
% use microtype if available
\IfFileExists{microtype.sty}{%
\usepackage{microtype}
\UseMicrotypeSet[protrusion]{basicmath} % disable protrusion for tt fonts
}{}
\usepackage[margin=1in]{geometry}
\usepackage{hyperref}
\hypersetup{unicode=true,
            pdftitle={Stochastic Process Final Project},
            pdfauthor={Wu, Tung Cheng},
            pdfborder={0 0 0},
            breaklinks=true}
\urlstyle{same}  % don't use monospace font for urls
\usepackage{graphicx,grffile}
\makeatletter
\def\maxwidth{\ifdim\Gin@nat@width>\linewidth\linewidth\else\Gin@nat@width\fi}
\def\maxheight{\ifdim\Gin@nat@height>\textheight\textheight\else\Gin@nat@height\fi}
\makeatother
% Scale images if necessary, so that they will not overflow the page
% margins by default, and it is still possible to overwrite the defaults
% using explicit options in \includegraphics[width, height, ...]{}
\setkeys{Gin}{width=\maxwidth,height=\maxheight,keepaspectratio}
\IfFileExists{parskip.sty}{%
\usepackage{parskip}
}{% else
\setlength{\parindent}{0pt}
\setlength{\parskip}{6pt plus 2pt minus 1pt}
}
\setlength{\emergencystretch}{3em}  % prevent overfull lines
\providecommand{\tightlist}{%
  \setlength{\itemsep}{0pt}\setlength{\parskip}{0pt}}
\setcounter{secnumdepth}{0}
% Redefines (sub)paragraphs to behave more like sections
\ifx\paragraph\undefined\else
\let\oldparagraph\paragraph
\renewcommand{\paragraph}[1]{\oldparagraph{#1}\mbox{}}
\fi
\ifx\subparagraph\undefined\else
\let\oldsubparagraph\subparagraph
\renewcommand{\subparagraph}[1]{\oldsubparagraph{#1}\mbox{}}
\fi

%%% Use protect on footnotes to avoid problems with footnotes in titles
\let\rmarkdownfootnote\footnote%
\def\footnote{\protect\rmarkdownfootnote}

%%% Change title format to be more compact
\usepackage{titling}

% Create subtitle command for use in maketitle
\newcommand{\subtitle}[1]{
  \posttitle{
    \begin{center}\large#1\end{center}
    }
}

\setlength{\droptitle}{-2em}

  \title{Stochastic Process Final Project}
    \pretitle{\vspace{\droptitle}\centering\huge}
  \posttitle{\par}
  \subtitle{Testing robustness of centrel limit theroy}
  \author{Wu, Tung Cheng}
    \preauthor{\centering\large\emph}
  \postauthor{\par}
    \date{}
    \predate{}\postdate{}
  
\usepackage{xeCJK}
\setCJKmainfont{標楷體}

\begin{document}
\maketitle

\section{一、簡介}

中央極限定理有許多不同的版本,根據 wikipedia
的說明:「大量互相獨立的隨機變數之均值,經適當的標準化後,依分布收斂於常態。」{[}1{]}。這個定義本身有許多可以延伸的問題,例如:

\begin{enumerate}
\def\labelenumi{\arabic{enumi}.}
\tightlist
\item
  何謂「適當的標準化」?
\item
  任意的隨機變數都可以成立嗎?
\item
  隨機變數的均質可以成立,那其他的描述統計量(最大值、標準差\ldots{})可以成立嗎?
\item
  是否有一個定值,可以稱為大量?
\end{enumerate}

本文將藉由統計語言R實作,嘗試回答上述問題,並以圖像呈現觀察到的結果。

\section{二、背景回顧}

最初始的中央極限定理版本,是法國數學家 Abraham de Moivre 於 1738
年在其著作
\href{https://books.google.com.tw/books?id=PII_AAAAcAAJ\&hl=zh-TW\&pg=PA235\#v=onepage\&q\&f=false}{The
Doctrine of Chances} 中所提出{[}2{]}。說明進行無限次獨立白努利測試
(Bernoulli trials) 所形成的二項分布,最終其機率質量函數(probability mass
function) 會近似於平均數為 \(np\)、標準差為 \(\sqrt{np(1-p)}\)
的常態分布,這個發現於 1810 年由 Pierre-Simon Laplace
使用特徵函數的方式證明{[}3{]}。其後的數學家們,如:Poisson, Cauchy,
Dirichlet, Lindeberg
也都對這個理論,提出擴展與相關的證明,使得後世的統計學得以對樣本有常態如此方便的假設。

\section{三、驗證問題}

\hypertarget{-}{%
\subsection{(一) 常態檢測工具}\label{-}}

我們要檢驗一個分布是否符合常態分布,因此需要檢驗的方法。在統計檢定上有許多檢測常態的方法,如
Kolmogorov-Smirnov test、The Anderson-Darling test、Shapiro-Wilks
test。在 Asghar Ghasemi 與 Saleh Zahediasl 的研究中{[}4{]},指出
Shapiro-Wilk test 是最穩定的方法,因此本文之常態檢定,將使用
Shapiro-Wilk test 進行。我們可以藉由檢定結果中的 p-value
判斷樣本是否為常態,若 p-value \(> 0.05\)
,表示樣本為常態分布。下面提供一個範例:

\begin{verbatim}
## 
##  Shapiro-Wilk normality test
## 
## data:  norm_dta
## W = 0.99965, p-value = 0.5652
\end{verbatim}

再提供一個樣本為 Poisson 的範例(即不符合常態分布)

\begin{verbatim}
## 
##  Shapiro-Wilk normality test
## 
## data:  poi_dta
## W = 0.94902, p-value < 2.2e-16
\end{verbatim}

除了 Shapiro-Wilk test 外,在實務上除了繪製機率密度函數圖,也常繪製 Q--Q
(quantile-quantile) plot,將手上之樣本與常態資料比較。因此在呈現
Shapiro-Wilk 檢定結果的同時,本文也將呈現 Q-Q plot。若 Q-Q plot
上的資料點落在紅線上,則表示樣本符合常態分布,上述符合常態配分之樣本的
Q-Q plot 如下:

\includegraphics[width=0.5\linewidth]{FinalProject_files/figure-latex/unnamed-chunk-3-1}
\includegraphics[width=0.5\linewidth]{FinalProject_files/figure-latex/unnamed-chunk-3-2}

樣本為 Poisson 的 Q-Q plot 如下:

\includegraphics[width=0.5\linewidth]{FinalProject_files/figure-latex/unnamed-chunk-4-1}
\includegraphics[width=0.5\linewidth]{FinalProject_files/figure-latex/unnamed-chunk-4-2}

\hypertarget{-}{%
\subsection{(二) 適當的標準化}\label{-}}

首先,我想先討論「適當的標準化」,因為會影響後續的實驗流程。測試:

\begin{enumerate}
\def\labelenumi{\arabic{enumi}.}
\tightlist
\item
  不標準化
\item
  \(\frac{(x-\mu_x)}{\sigma_x}\)
\end{enumerate}

我們使用 R 的 function 製作 \(N(15, 21)\) 的常態分布 5000
個,並取其各自的平均數。

\begin{verbatim}
## 
##  Shapiro-Wilk normality test
## 
## data:  mean_dta
## W = 0.9997, p-value = 0.6973
\end{verbatim}

\includegraphics[width=0.5\linewidth]{FinalProject_files/figure-latex/unnamed-chunk-5-1}
\includegraphics[width=0.5\linewidth]{FinalProject_files/figure-latex/unnamed-chunk-5-2}

我們針對尚未標準化的樣本進行檢驗,從 Shapiro-Wilk test
的結果,我們可以看到 p \textgreater{} 0.05。從 Q-Q plot
中也可以觀察到,數據點幾乎都在紅線上,表示與常態分布相似。這與 wiki
上的說明不相符,但根據老師上課的說明,經過標準化的樣本是符合標準常態分布。

接著我們來看,經過 \(\frac{(x-\mu_x)}{\sigma_x}\) 標準化的結果

\begin{verbatim}
## 
##  Shapiro-Wilk normality test
## 
## data:  dta.scale
## W = 0.9997, p-value = 0.6973
\end{verbatim}

\includegraphics[width=0.5\linewidth]{FinalProject_files/figure-latex/unnamed-chunk-6-1}
\includegraphics[width=0.5\linewidth]{FinalProject_files/figure-latex/unnamed-chunk-6-2}

從檢定的結果上來說並沒有差異,但觀察兩張圖的縱軸,可以發現數值與刻度寬度不同,表示兩樣本的平均與標準差不同。總結來說,就我們的實驗結果顯示,不標準化不影響是否為常態分布,但影響最後收斂於哪個常態。為避免有本實驗未觀察到的情況,往後實驗仍會進行標準化。

\hypertarget{-}{%
\subsection{(三) 任意的隨機變數}\label{-}}

接著我們討論不同隨機變數的平均數,是否也會收斂於常態分布,以下測試一些上課提過的分布:

\begin{enumerate}
\def\labelenumi{\arabic{enumi}.}
\tightlist
\item
  Binomial distribution (set probability = 0.5)
\end{enumerate}

\begin{verbatim}
## 
##  Shapiro-Wilk normality test
## 
## data:  mean_dta.binom
## W = 0.99977, p-value = 0.8821
\end{verbatim}

\includegraphics[width=0.5\linewidth]{FinalProject_files/figure-latex/unnamed-chunk-7-1}
\includegraphics[width=0.5\linewidth]{FinalProject_files/figure-latex/unnamed-chunk-7-2}

\begin{enumerate}
\def\labelenumi{\arabic{enumi}.}
\setcounter{enumi}{1}
\tightlist
\item
  Poisson distribution (set \(\lambda = 3\))
\end{enumerate}

\begin{verbatim}
## 
##  Shapiro-Wilk normality test
## 
## data:  mean_dta.poi
## W = 0.99947, p-value = 0.1668
\end{verbatim}

\includegraphics[width=0.5\linewidth]{FinalProject_files/figure-latex/unnamed-chunk-8-1}
\includegraphics[width=0.5\linewidth]{FinalProject_files/figure-latex/unnamed-chunk-8-2}

\begin{enumerate}
\def\labelenumi{\arabic{enumi}.}
\setcounter{enumi}{2}
\tightlist
\item
  Uniform distribution (set Uniform distribution \(\in [0,1]\))
\end{enumerate}

\begin{verbatim}
## 
##  Shapiro-Wilk normality test
## 
## data:  mean_dta.uni
## W = 0.9995, p-value = 0.2066
\end{verbatim}

\includegraphics[width=0.5\linewidth]{FinalProject_files/figure-latex/unnamed-chunk-9-1}
\includegraphics[width=0.5\linewidth]{FinalProject_files/figure-latex/unnamed-chunk-9-2}

\begin{enumerate}
\def\labelenumi{\arabic{enumi}.}
\setcounter{enumi}{3}
\tightlist
\item
  Exponential distribution (set \(\lambda = 3\))
\end{enumerate}

\begin{verbatim}
## 
##  Shapiro-Wilk normality test
## 
## data:  mean_dta.exp
## W = 0.99972, p-value = 0.7763
\end{verbatim}

\includegraphics[width=0.5\linewidth]{FinalProject_files/figure-latex/unnamed-chunk-10-1}
\includegraphics[width=0.5\linewidth]{FinalProject_files/figure-latex/unnamed-chunk-10-2}

由上述的檢定與 Q-Q plot
顯示,隨機變數的種類,不影響其平均值是否收斂於常態分布。總結來說,在上課提到的四個隨機變數,經過測試後發現
5000 個樣本平均值,會收斂於常態分布。

\hypertarget{-}{%
\subsection{(四) 其他的描述統計量}\label{-}}

Wikipedia
說明的中央極限定理,隨機變數的平均數會收斂於常態分布,以下測試其他的描述統計量(ex:
標準差、最大值、最小值、偏態(skewness, 三階動差)、峰度(kurtosis,
四階動差)),使用前述的 \(N(15, 21)\) 分布進行測試。

\begin{enumerate}
\def\labelenumi{\arabic{enumi}.}
\tightlist
\item
  標準差
\end{enumerate}

\begin{verbatim}
## 
##  Shapiro-Wilk normality test
## 
## data:  sd_dta
## W = 0.99974, p-value = 0.8158
\end{verbatim}

\includegraphics[width=0.5\linewidth]{FinalProject_files/figure-latex/unnamed-chunk-11-1}
\includegraphics[width=0.5\linewidth]{FinalProject_files/figure-latex/unnamed-chunk-11-2}

\begin{enumerate}
\def\labelenumi{\arabic{enumi}.}
\setcounter{enumi}{1}
\tightlist
\item
  最大值
\end{enumerate}

\begin{verbatim}
## 
##  Shapiro-Wilk normality test
## 
## data:  max_dta
## W = 0.9623, p-value < 2.2e-16
\end{verbatim}

\includegraphics[width=0.5\linewidth]{FinalProject_files/figure-latex/unnamed-chunk-12-1}
\includegraphics[width=0.5\linewidth]{FinalProject_files/figure-latex/unnamed-chunk-12-2}

\begin{enumerate}
\def\labelenumi{\arabic{enumi}.}
\setcounter{enumi}{2}
\tightlist
\item
  最小值
\end{enumerate}

\begin{verbatim}
## 
##  Shapiro-Wilk normality test
## 
## data:  min_dta
## W = 0.96835, p-value < 2.2e-16
\end{verbatim}

\includegraphics[width=0.5\linewidth]{FinalProject_files/figure-latex/unnamed-chunk-13-1}
\includegraphics[width=0.5\linewidth]{FinalProject_files/figure-latex/unnamed-chunk-13-2}

\begin{enumerate}
\def\labelenumi{\arabic{enumi}.}
\setcounter{enumi}{3}
\tightlist
\item
  偏態
\end{enumerate}

\begin{verbatim}
## 
##  Shapiro-Wilk normality test
## 
## data:  skew_dta
## W = 0.99971, p-value = 0.7493
\end{verbatim}

\includegraphics[width=0.5\linewidth]{FinalProject_files/figure-latex/unnamed-chunk-14-1}
\includegraphics[width=0.5\linewidth]{FinalProject_files/figure-latex/unnamed-chunk-14-2}

\begin{enumerate}
\def\labelenumi{\arabic{enumi}.}
\setcounter{enumi}{4}
\tightlist
\item
  峰度
\end{enumerate}

\begin{verbatim}
## 
##  Shapiro-Wilk normality test
## 
## data:  kurt_dta
## W = 0.99798, p-value = 4.202e-06
\end{verbatim}

\includegraphics[width=0.5\linewidth]{FinalProject_files/figure-latex/unnamed-chunk-15-1}
\includegraphics[width=0.5\linewidth]{FinalProject_files/figure-latex/unnamed-chunk-15-2}

透過上面的測試,我們發現改成其他描述統計量時,只有標準差與偏態
(即二階與三階動差) 會收斂於常態。提出以下兩種可能性:

\begin{enumerate}
\def\labelenumi{\arabic{enumi}.}
\tightlist
\item
  所有的描述統計量都會收斂於常態分布,只是目前 5000
  個隨機變數的描述統計量,可能不夠「大量」,因此沒有收斂於常態分布
\item
  只有動差類的描述統計量,會收斂於常態分布,隨著階層越高,需要的「大量」也越大,因此本次測試的峰度並沒有收斂。
\end{enumerate}

上述的可能性可以歸納成兩個測試項目:

\begin{enumerate}
\def\labelenumi{\arabic{enumi}.}
\tightlist
\item
  增加隨機變數的量
\end{enumerate}

然而本文中使用的 Shapiro-Wilk 最多接受 5000
筆資料。但更換檢定方法,無法確定效果來自「更換檢定方法」還是「增加樣本量」。由於時間因素,本文增加樣本數繼續測試。

\begin{enumerate}
\def\labelenumi{\arabic{enumi}.}
\setcounter{enumi}{1}
\tightlist
\item
  使用其他描述統計量測試 (下方使用中位數測試)
\end{enumerate}

\begin{verbatim}
## 
##  Shapiro-Wilk normality test
## 
## data:  median_dta
## W = 0.99972, p-value = 0.7681
\end{verbatim}

\includegraphics[width=0.5\linewidth]{FinalProject_files/figure-latex/unnamed-chunk-16-1}
\includegraphics[width=0.5\linewidth]{FinalProject_files/figure-latex/unnamed-chunk-16-2}

使用中位數測試後發現,並非只有動差類的描述統計量,會收斂於常態分布。整體來說,更改樣本的描述統計值時,經過測試結果發現「隨機變數的中位數、標準差、偏態,會收斂於常態分布」。其他的描述統計量,由於
Shapiro-Wilk test
的限制,不確定是不會收斂,還是樣本不夠大量,因此沒有收斂於常態分配。此處的隨機變數是常態分布,並未測試其他分布的情況。

\hypertarget{-}{%
\subsection{(五) 大量有多大}\label{-}}

最後測試需要多少樣本數才能稱為大量,由於在極小樣本時,p-value
的參考價值有限,故本小節主要觀察機率密度函數圖與 Q-Q
plot。以下測試不同數量隨機變數的平均數,會收斂於常態分布。

\begin{enumerate}
\def\labelenumi{\arabic{enumi}.}
\tightlist
\item
  \(N = 3\)
\end{enumerate}

\begin{verbatim}
## 
##  Shapiro-Wilk normality test
## 
## data:  N3_dta
## W = 0.93208, p-value = 0.4965
\end{verbatim}

\includegraphics[width=0.5\linewidth]{FinalProject_files/figure-latex/unnamed-chunk-17-1}
\includegraphics[width=0.5\linewidth]{FinalProject_files/figure-latex/unnamed-chunk-17-2}

\begin{enumerate}
\def\labelenumi{\arabic{enumi}.}
\setcounter{enumi}{1}
\tightlist
\item
  \(N = 5\)
\end{enumerate}

\begin{verbatim}
## 
##  Shapiro-Wilk normality test
## 
## data:  N5_dta
## W = 0.98944, p-value = 0.9777
\end{verbatim}

\includegraphics[width=0.5\linewidth]{FinalProject_files/figure-latex/unnamed-chunk-18-1}
\includegraphics[width=0.5\linewidth]{FinalProject_files/figure-latex/unnamed-chunk-18-2}

\begin{enumerate}
\def\labelenumi{\arabic{enumi}.}
\setcounter{enumi}{2}
\tightlist
\item
  \(N = 10\)
\end{enumerate}

\begin{verbatim}
## 
##  Shapiro-Wilk normality test
## 
## data:  N10_dta
## W = 0.96593, p-value = 0.8508
\end{verbatim}

\includegraphics[width=0.5\linewidth]{FinalProject_files/figure-latex/unnamed-chunk-19-1}
\includegraphics[width=0.5\linewidth]{FinalProject_files/figure-latex/unnamed-chunk-19-2}

\begin{enumerate}
\def\labelenumi{\arabic{enumi}.}
\setcounter{enumi}{3}
\tightlist
\item
  \(N = 20\)
\end{enumerate}

\begin{verbatim}
## 
##  Shapiro-Wilk normality test
## 
## data:  N20_dta
## W = 0.98333, p-value = 0.9696
\end{verbatim}

\includegraphics[width=0.5\linewidth]{FinalProject_files/figure-latex/unnamed-chunk-20-1}
\includegraphics[width=0.5\linewidth]{FinalProject_files/figure-latex/unnamed-chunk-20-2}

\begin{enumerate}
\def\labelenumi{\arabic{enumi}.}
\setcounter{enumi}{4}
\tightlist
\item
  \(N = 30\)
\end{enumerate}

\begin{verbatim}
## 
##  Shapiro-Wilk normality test
## 
## data:  N30_dta
## W = 0.96898, p-value = 0.5116
\end{verbatim}

\includegraphics[width=0.5\linewidth]{FinalProject_files/figure-latex/unnamed-chunk-21-1}
\includegraphics[width=0.5\linewidth]{FinalProject_files/figure-latex/unnamed-chunk-21-2}

\begin{enumerate}
\def\labelenumi{\arabic{enumi}.}
\setcounter{enumi}{4}
\tightlist
\item
  \(N = 50\)
\end{enumerate}

\begin{verbatim}
## 
##  Shapiro-Wilk normality test
## 
## data:  N50_dta
## W = 0.98596, p-value = 0.8118
\end{verbatim}

\includegraphics[width=0.5\linewidth]{FinalProject_files/figure-latex/unnamed-chunk-22-1}
\includegraphics[width=0.5\linewidth]{FinalProject_files/figure-latex/unnamed-chunk-22-2}

\section{四、總結}

經過本文的一系列實驗,關於「大量互相獨立的隨機變數之均值,經適當的標準化後,依分布收斂於常態。」我們提出了四個問題並發現,中央極限定理比想像中的還要穩定。不論是替換隨機變數,或是計算不同的隨機變數描述統計值,甚至是沒有標準化,觀察到結果依然收斂於常態分布。

不過本文也有些不完整之處,個人認為最明顯的問題在於檢定常態的方法。本文使用統計檢定的方式,並直接使用
p-value 來判斷結果,這其實是一個蠻危險的作法。p-value
只是眾多指標中的一個指標,由於使用簡單與教學方便,因此在許多檢定上常被使用,但不表示
p-value 可以當作唯一的指標。(ex: 重複 100 次實驗,可能會得到 94 次
p-value \textless{} 0.05 的結果,當觀察到另外的 6
次時,直接下結論就錯了。)

期望若有學弟妹想要做相關的 final
project,可以使用其他的方法來判斷是否為常態分布。

\section{五、心得}

從大一以來就一直聽聞中央極限定理的神奇事蹟,但也就停留在最基本的認識,很開心透過這次的
Final Project
能與這位認識已久的朋友,有更深入的認識。若學弟妹有需要,還請老師將這份報告分享給他們,感謝老師這學期的指導。

\hypertarget{reference}{%
\section*{Reference}\label{reference}}
\addcontentsline{toc}{section}{Reference}

\hypertarget{refs}{}
\leavevmode\hypertarget{ref-wiki:xxx}{}%
{[}1{]} wikipedia, ``中央極限定理 - wikipedia, the free encyclopedia.''
2018.

\leavevmode\hypertarget{ref-de1738doctrine}{}%
{[}2{]} A. de Moivre, \emph{The doctrine of chances}. Woodfall, 1738,
pp. 235--243.

\leavevmode\hypertarget{ref-Fischer2011}{}%
{[}3{]} H. Fischer, ``The central limit theorem from laplace to cauchy:
Changes in stochastic objectives and in analytical methods,'' in \emph{A
history of the central limit theorem: From classical to modern
probability theory}, New York, NY: Springer New York, 2011, pp. 17--74.

\leavevmode\hypertarget{ref-ghasemi2012normality}{}%
{[}4{]} A. Ghasemi and S. Zahediasl, ``Normality tests for statistical
analysis: A guide for non-statisticians,'' \emph{International journal
of endocrinology and metabolism}, vol. 10, no. 2, p. 486, 2012.


\end{document}
